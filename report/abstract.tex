\begin{abstract}
The explosion of mobile devices have led to tremendous demand for network services and bandwidth. This has necessitated the development of intelligent networks which allows dynamic reconfiguration based upon channel conditions, sprectrum availability and other user defined conditions. This requires a cross layer visibility into the protocol stack and a centralized view of the network for optimal decision making. In order to provide dynamic adjustment and reconfigurability of MAC protocols, we present the PMAC framework. The framework focusses on defining MAC layer abstractions and providing interfaces for event driven programmability using Software Defined Networking (SDN) principles. It also provides a mechanism for the devices to interact with the centralized controller by piggybacking protocol messages on beacon frames to avoid creating a new messaging system. These messages are processed by the Access Points before sending it to the centralized controller through SDN control channel. We assume that NIC vendors provide the stated APIs and allow different MAC protocols to be loaded at runtime. As part of this paper we extend the open source Glomosim network simulator to support dynmaic switching of protocols and showcase a proof-of concept application of changing MAC protcols according to hidden terminal and exposed terminal scenarios. 
Dynamic reconfiguration and network programmability are active research areas. State of the art solutions use the Software Defined Networking (SDN) paradigm to provide basic data plane abstractions and programming interfaces for control and management of these abstractions. This approach provides the benefit of control and data plane separation, but it is mainly limited to wired networks. Currently, SDN technologies do not provide appropriate abstractions to support constantly evolving wireless protocols. On the other hand, the Software Defined Radio (SDR) paradigm enables complex signal processing functionality to be implemented efficiently in software, instead of on specialized hardware. However, SDR does not cater to the demand for adaptive radio network management with respect to changing channel conditions and policies. 
%Current approaches focus upon specific protocol implementations and do not provide a generic and principled framework for developing network applications. 
To address this, we present \crossflow, a principled approach for application development in radio networks. \crossflow defines several fundamental radio port abstractions and an interface to manipulate them. The framework provides a flexible and modular cross-layer architecture using the principles of SDR and a mechanism for centralized control using the principles of SDN. The main feature of the \crossflow architecture is that it provides a protocol independent framework for application development in wireless radio networks. We validate our design using proof-of-concept applications, namely, adaptive modulation, frequency hopping, and cognitive radio. Our results indicate that our framework is efficient, flexible, and can be used for a variety of applications.
\end{abstract}
