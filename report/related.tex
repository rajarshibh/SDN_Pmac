\section{Related Work}
\label{sec:related}

The idea of providing a flexible MAC was implemented by various vendors in the form of \textit{soft-MAC}\cite{softmac}, where most of the time critical operations like synchronization, transmission and reception operations were deployed in hardware, while the other non-critical operations like admission control and rate control were performed in software. However this technique does not provide full control of MAC layer properties to the programmer. Some solutions like the Software Defined Radio framework provided by \cite{gnuradio} try to provide flexibity but it suffers from throughput loss. Other Software Defined Radio initiatives like \cite{sora} try to provide a hardware-like throughput, but suffers from inflexibity due to its complex architecture and requires significant expertise for operation. 

\cite{macproc} provides a high degree of programmability of MAC layer as it represents each MAC protocol as a finite state machine and provides a set of interfaces to manipulate the state of the machine. In fact, we use the paper's interfaces to define our own MAC abstraction layer. The paper \cite{maclet} extends \cite{macproc} by allowing the installation of a \textit{on-the-fly} full MAC protocol logic, rather than a set of parameter settings. The main difference between these works and our work is that we focus on providing an abstract layer for the interfaces exposed by the NIC vendor in a consistent manner, so as to avoid ad-hoc approaches. This allows applications to be developed in a much easier way, and provides a clean separation between application logic and implementation. The previous papers mainly focus on mechanisms to provide these interfaces, which is orthogonal to our work. The exisiting approaches use details which are too low level and need sophisticated handling by the application to implement an event based asynchronous programming model.  
