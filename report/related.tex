\section{Background and Related Work}
\label{sec:related}
\textbf{Software Defined Networking.}
Network reconfigurability is a major challenge in the networking industry. The explosion of mobile devices and cloud services have necessitated the need for on-demand installation of services and reconfiguration of flow rules according to changing traffic patterns. In addition, network elements like routers and switches have their own unique interfaces and as such management of network components is a source of concern for network application developers. As the network grows, this complexity increases exponentially and rolling out new services becomes a tedious and complicated process.

Software Defined Networking (SDN) is an architecture which tries to address these challenges by decoupling the control and forwarding functions. This enforces abstraction of underlying implementation and enables applications or network services to be developed using the abstractions. This simple and elegant design also provides applications a centralized view of the network. As a result, it has sparked tremendous research interest in providing a scalable, secure and programmatic approach towards the challenges discussed above. While SDN is a revolutionary approach, it is still mainly geared towards wired networks. 
%The wireless networks have not received much attention in this regard. 
Through our previous work, \aetherflow \cite{aetherflow}, we tried to provide a protocol independent approach for bringing wireless into the SDN model. In this paper, we go a step further and try to provide a mechanism for dynamic radio resource management to obtain true network visibility in a heterogeneous network.     

\textbf{GNU Radio framework.}
GNU Radio \cite{gnuradio} is a free and open-source framework that provides signal processing functionality to implement SDRs. The main constituents of the framework are basic blocks which perform distinct signal processing functions. GNU Radio enables the composition of these blocks to synthesize new radio functionality on general purpose hardware, but it is not suitable for developing applications to control a network of SDRs. This is because each block exposes its own set of interfaces which does not scale with increasing numbers of radios in the network. In this paper, we provide uniform interfaces to control and manage these processing block abstractions, so that an application developer does not need to handle every block's unique interface characteristics.

Aside from GNU Radio, the idea of providing a programmable wireless data plane has been implemented in~\cite{atomix} and~\cite{openradio}. Both these papers provide modular blocks and focus on real time guarantees for processing signals. But like GNU Radio, they do not provide any logical interface to control a network of such programmable devices. We choose GNU Radio in our design because it is flexible, open-source, and widely used. The paper~\cite{softran} deals with centralized control of devices but it focuses mainly on LTE networks. Our paper is orthogonal to these works as we provide a mechanism for centralized control while making the exposed interfaces protocol independent.
The combination of SDRs and SDN has recently been used in a variety of contexts~\cite{cho2014integration, sun2015integrating,mancuso2014prototyping,corbett2014countering}. In~\cite{mancuso2014prototyping}, SDR and SDR are used to create a testbed for LTE technologies while~\cite{cho2014integration, sun2015integrating} focus on integration of SDN and SDR for 4G/5G technology. In~\cite{gupta2015labview}, an SDR model for management of interference in dense heterogeneous networks is proposed while~\cite{corbett2014countering} developed a jamming architecture using SDN and SDR principles. These papers provide distinct solutions for various scenarios but do not provide a generic framework for handling various protocols in a principled manner.
