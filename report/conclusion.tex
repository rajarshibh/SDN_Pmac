\section{Conclusion and Future Work}
\label{sec:conclusion}
In this paper, we presented the \pmac framework for enabling a programmatic MAC through a SDN controller. We assume that the NIC vendors provide the set of primitives for reconfiguring the MAC including switching between MAC protocols. We build upon these primitives to build an MAC abstraction layer which allows flexible and real-time (re)configuration of the MAC layer by hiding the low level API details. We validate our approach by providing a proof-of-concept application which showcases the utility of dynamic MAC reconfiguration in a simulated environment. In this way, we also extend the capability of GloMoSim simulator to support MAC reconfigurability.

One of the real tasks that needs to be accomplished is implementing \pmac framework on a real device. To do this, we must either reflash commercially available commodity hardware or implement it on a Software Defined Radio framework like GNURadio. Once we have the exposed APIs, the firmware of APs must be loaded with the agents to communicate with the OpenFlow enabled SDN controller. We also plan to reflash the fireware of AP with OpenWRT \cite{openwrt} for flexibility. In order to reduce latency between controller and AP, the AP must be programmable and the OpenFlow table loaded in AP must be extended to include core MAC layer functionality.
