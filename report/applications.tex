\section{Applications}

\subsection{Frequency Hopping}
Frequency hopping is a technique of transmitting radio signals by spreading the signal over a sequence of changing frequencies. It has tremendous application in military as it is used against jamming and for protecting against unauthorized eavesdropping. For implementation, the receiver of the signal must be aware of the sequence of frequencies so that it can tune into the appropriate channel. This requires synchronization between the transmitter and the receiver. In \crossflow, we implement this application easily as only the controller needs to be aware about the predetermined sequence. This sequence can even be dynamic according to the channel conditions and policies. The controller simply issues \emph{GNU-CONFIG-FREQ} command with desired frequency and pushes this configuration to the device. The \texttt{ofsoftswitch} receives this command and forwards it to the GNU Radio domain. The centralized \texttt{\crossflow Hub} inside the GNU Radio domain processes this request and issues appropriate commands to the  USRP Controller, which ultimately signals the USRP block to tune into the requested frequency. Figure~\ref{fig:freq} shows the experimental results where the sequence of changing frequencies is 910, 915 and 920 MHz and is changed every 5 seconds with BPSK fixed modulation scheme.   

\subsection{Adaptive Modulation}
Adpative Modulation is a technique where the modulation is changed according to the conditions of the channel. There are various estimators which are used for obtaining channel quality. These can be Signal-to-noise ratio (SNR), Bit error rate (BER) and other environment specific estimators. For illustration, we assume a fixed sequence for changing the modulation schemes every 5 seconds. Similar to the \emph{frequency hopping} application, the controller issues the \emph{GNU-CONFIG-MOD} command with the appropriate modulation scheme like BPSK, QPSK etc and forwards the request to the device. The request ultimately reaches the MOD controller, which is a multiplexer block that selects the requested modulation scheme. Figure~\ref{fig:mod} shows the results for changing the modulation scheme between BPSK, QPSK and 8PSK every 5 seconds, keeping a fixed carrier frequency of 910 MHz.   

