\section{Introduction}
\label{sec:intro}
Wireless communication has become as an integral part of our lives. This is mainly due to the rise of smart phones, tablets and other hand held devices in recent years. As a result, network operators must now support not only greater bandwidth but also on-demand service requirements of users. In order to keep pace with rapidly changing network requirements, the network should provide abstract interfaces so that it is programmable and provide mechanisms to take globally optimal decisions.

In this paper, we focus on making the Medium Access Control (MAC) layer of the network protocol stack programmable. MAC plays a critical role in co-ordinating access of the shared wireless medium. A sub-optimal MAC can significantly degrade the throughput achieved by individual devices and can lead to poor user experience of applications. In order to serve greater service requriements, Network Interfaces Card (NIC)vendors have typically implemented  customized versions of predominant MAC protocols like 802.11, and as a result the network is fragmented at the MAC layer. The need of the hour, from the point of view of network operators, is that NIC vendors provide a consistent set of interfaces so that the MAC layer can be configured, queried and registered for specific events in response to changing channel or business requirements.

For the purpose of supporting a programmatic MAC and to take globally optimal decisions at the MAC layer, we present the \pmac framework. The framework builds upon the interfaces described in \cite{macproc} to abstract away the implementation specifc features of MAC layer, thereby enabling applications to be written without being worried about low-level details of specific protocols. This decoupling of functionality essentially allows a centralzied controller to co-ordinate functions of MAC layer for all the devices in the network. This follows the traditional model of Software Defined Networking (SDN) of separating the control and data planes thereby allowing remote configuration of devices through applications. The applications can be written to implement cross-layer functionality for devices' networking stack using the global view provided by the controller. This approach is a paradigm shift from the traditional distributed coordination functionality of MAC protocols to a more centralized policy oriented  functionality. \pmac retains the distributed nature of MAC protocols for time sensitive operations like synchronization, channel estimation etc, but also allows policies to be defined dynamically based upon changing channel conditions and density of devices in the network.  

While the SDN approach provides significant benefits, it is mainly geared towards wireline devices. The de-facto SDN protocol Openflow \cite{openflow} does not contain any support for wireless protocols. We use the extension framework suggested in \cite{aetherflow} to incorporate MAC layer functionality into OpenFlow so that a controller has visibility into the MAC layer of devices. The \pmac framework envisions an architecture where the SDN controller interacts with the devices through Access Points (APs) in infrastructure mode. It also defines a protocol for SDN configuration control messages between devices and APs. This is necessary to achieve synchronization between devices for MAC reconfiguration and to avoid inconsistencies in application of MAC configuration.  


In the current implementation, we use GloMoSim network simulator \cite{glomosim} to showcase two scenarios: hidden terminal problem and exposed terminal problem, where dynamic reconfiguration of MAC protocols leads to better throughput experienced by the devices. This proof-of-concept implementation involves an application which interacts with a MAC abstraction layer implemented in GloMoSim. This abstract layer in turn interacts with the simulated nodes for implementing functionality requested by the application. The application normally uses the controller API to interact with devices, but in this initial phase we do not consider a SDN controller, and hence the application directly interacts with the simulated nodes. We also discuss the integration of \pmac framework with SDN controller in subsequent sections. 


Our contributions can be summarized as follows.

\begin{itemize}
\item We propose a framework that abstracts the functionality of the MAC layer.

\item We extend the SDN model with message extensions to provide support for MAC layer.

\item We also describe a protocol for applying SDN MAC layer configuration messages from controller.  

\item We provide sample a proof-of-concept application which solves the problems of hidden terminal and exposed terminal through dynamic reconfiguration of MAC.
\end{itemize}

The rest of the paper is organized as follows. In Section~\ref{sec:related} we review the related work done in this area. Section~\ref{sec:architecture} describes the \pmac architecture with its SDN extensions. Section~\ref{sec:evaluation} describes the proof-of-concept implementation of our framwork and the results obtained using the GloMoSim simulator. Section~\ref{sec:conclusion} concludes the paper and discusses future work.
